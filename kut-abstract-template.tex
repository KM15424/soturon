%% 内容梗概

%% プリアンブル %%%%%%%%%%%%%%%%%%%%%%%%%%%%%%%%%%%%%%%%%%%%%%%%%%%%%%%%
\documentclass[a4j]{jarticle}

\usepackage{kut-abstract}
\usepackage{multirow}
\usepackage[dvipdfmx]{graphicx}
\usepackage{url}
\usepackage[deluxe]{otf}
\usepackage{here}
\usepackage{pdfpages}

\usepackage{algorithmic}
\usepackage{algorithm}

%% 表題 %%%%%%%%%%%%%%%%%%%%%%%%%%%%%%%%%%%%%%%%%%%%%%%%%%%%%%%%%%%%%%%%
\ScInfo         %% 情報学群生以外の場合はコメントアウトする

\Bachelor	%% 学士学位論文(卒業研究)梗概の場合
%\Project	%% プロジェクト研究報告書梗概の場合
%\Seminar	%% 特別研究セミナー課題研究報告書梗概の場合
%\Master	%% 修士学位論文(情報システム工学コース)梗概の場合
%\Doctorate	%% 博士学位論文(情報システム工学コース)梗概の場合
%\English	%% 英語の場合

\years{2024}
\Eyears{}	%% 英語の場合
\title{「系列」を用いた麻雀上級者の戦略抽出}
\Etitle{}	%% 修士学位論文梗概または英語の場合
\idnumber{1240314}
\author{黒田~~将司}
\Eauthor{}	%% 修士学位論文梗概または英語の場合
\affiliate{ゲーム情報学研究室}
\Eaffiliate{}	%% 修士学位論文梗概または英語の場合

%% 本文 %%%%%%%%%%%%%%%%%%%%%%%%%%%%%%%%%%%%%%%%%%%%%%%%%%%%%%%%%%%%%%%%
\begin{document}
\begin{Abstract}

 \section{はじめに}
 近年,様々なゲームにおいて AI の開発が盛んになっており,オセロや将棋などプロプレイヤを上回る強さの AI も出てきている.しかし,強い AI の持つ戦略は複雑なアルゴリズムなどからできており,人間にとって分かりずらく,理解することが難しい.
 
 そこで本研究では,主に戦略を考えることが難しいとされる麻雀において,上級者の打牌から人間に分かりやすい方法で戦略を見つけることを目的とする.

 \section{提案手法}
 従来の研究では,プレイヤの捨て牌の系列記録から,頻出パターン系列を抽出することで勝者と敗者の固有パターンを調べている\cite{sakaida}.その結果,麻雀で勝つためには攻めるのではなく守りながら戦うことの方が有効であるという結論を出している.本研究では,上級者プレイヤの打牌に「系列」という概念を用いることで,特定局面での戦略を抽出することを目指す.
 
 \subsection{系列の定義}
 本研究で定義する「系列」は,前川の研究\cite{maekawa}で定義されていたものであり,プレイヤの手牌の変化を表したモデルである.
 系列はタプル(前系列,次系列)により構成され,自摸局面から打牌後への手牌の変化を読み取ることができる.
 \begin{figure}[h]
  \setlength\abovecaptionskip{-1mm}
  \centering
    \includegraphics[scale=0.38]{keiretu2.pdf}
    \caption{自摸局面と打牌後の例}
    \label{keiretu_fig}
    \vspace{-1zh}
\end{figure}

 例として,図\ref{keiretu_fig}の状況では,形の変化した索子の部分に着目し,(1234,234)という系列を得ることができる.

\section{実験}
 系列のから上級者の戦略を抽出するための実験を行う.
 実験には,オンライン麻雀サイトである天鳳での鳳凰卓の牌譜 5000 半荘分と天鳳位の牌譜 5000 半荘分を利用する.また,系列の集計用の牌譜と,集計した系列と実際の牌譜で選ばれた打牌を比較するための牌譜として,鳳凰卓と天鳳位の牌譜それぞれ 5000 半荘分の内4500半荘を前者,500半荘を後者に利用する.また,天鳳位の戦略を抽出しやすくするために,鳳凰卓の牌譜からは天鳳位との実力差が大きいと思われる16段のプレイヤのみのデータを集計と比較に用いる.
 
\subsection{全体データとの比較実験}
 実験では,鳳凰卓の牌譜 4500 半荘分と天鳳位の牌譜 4500 半荘分を合わせた9000 半荘分の牌譜から系列データを集計する.集計した系列データと比較用の牌譜の打牌を比較することで系列データから取得できた打牌傾向と実際の打牌の一致率を測る.
 
\subsection{鳳凰卓,天鳳位データとの比較実験}
 実験では,鳳凰卓の牌譜 4500 半荘分と天鳳位の牌譜 4500 半荘分の牌譜それぞれで系列データを集計する.集計した系列データと比較用の牌譜の打牌を比較することで系列データから取得できた打牌傾向と実際の打牌の一致率を測る.

\section{実験結果}
 表\ref{tab:exp1_result}に全体データとの比較実験の結果を示す.
 両方の比較データ共に一致率が40\%の打牌を選択できており,あまり違いはなかった.
 
  \begin{table}[h]
   \vspace{-1.5zh}
  \begin{center}
   \caption{全体データとの比較実験結果}
   \label{tab:exp1_result}
    \vspace{0.3zh}
   \begin{tabular}{r||r} \hline \hline
	比較データ&一致率 \\ \hline
	天鳳位 & 41.4\% \\ \hline
	鳳凰卓16段&  40.5\% \\ \hline
   \end{tabular}
  \end{center}
 \end{table}
 \vspace{-2.0zh}
 表\ref{tab:exp2_result}に鳳凰卓,天鳳位データとの比較実験の結果を示す.
 すべての比較結果で約40\% の一致率で打牌を選択できており,あまり違いはなかった.両方の系列データで鳳凰卓16段との一致率のほうが低い結果となった.
 
  \begin{table}[h]
   \vspace{-1.5zh}
  \begin{center}
   \caption{鳳凰卓,天鳳位データとの比較実験結果}
   \label{tab:exp2_result}
    \vspace{0.3zh}
   \begin{tabular}{r||r|r} \hline \hline
	系列データ&天鳳位一致率&鳳凰卓16段一致率 \\ \hline
	天鳳位 & 40.4\%& 39.0\% \\ \hline
	鳳凰卓16段 & 41.2\%&  41.0\% \\ \hline
   \end{tabular}
  \end{center}
 \end{table}
 \vspace{-2.0zh}
 
\section{まとめ}
 今回の実験では,上級者のみにみられる戦略を抽出することはできなかった.今後の展望として,データを親のリーチ後のみなど特定条件に絞った比較を行うことで上級者のみにみられる戦略を抽出する予定だ.
 

%% 参考文献 %%%%%%%%%%%%%%%%%%%%%%%%%%%%%%%%%%%%%%%%%%%%%%%%%%%%%%%%%%%%
\begin{thebibliography}{99}
\bibitem{sakaida} 堺田寛一朗,川又泰介,松田源立,“シーケンシャルパターンマイニングを用いた麻雀の捨て牌の傾向分析”,情報処理学会第84回全国大会講演論文集,pp.404-405,2022.
\bibitem{maekawa}	前川幸輝,“系列情報を用いて人間の模倣を行う麻雀 AI の研究”,令和 4 年度 修士学位論文,2023.
\end{thebibliography}

\end{Abstract}
\end{document}
